\begin{longtable}{|>{\setlength{\hsize}{.7\hsize}}X|>{\setlength{\hsize}{.8\hsize}}X|>{\setlength{\hsize}{1.8\hsize}}X|>{\setlength{\hsize}{.7\hsize}}X|}
\hline	\rowcolor{red}
	Données& Description& Valeurs possibles& Mode\\
	\hline\hline
	\rowcolor{grisFonce}
	Name &Type de la trace technique&
	Prend la valeur :\begin{itemize}
		\item LogInfo
		\item LogWarning
		\item LogError
		\item LogSuccess
	\end{itemize} &Consultation\\
\hline\rowcolor{gris} Message&Un message ou un commentaire&&Consultation\\
	\hline\rowcolor{grisFonce}
	Message code& Identifiant du point de passage &
	\begin{minipage}[t]{\hsize}
			\begin{itemize}
			\item PFGE\_IN:Entrée d'un message, d'une requête ou d'un fichier dans HUBIC 
			\item PFGE\_OUT : Sortie d'un message,d'une requête ou d'un fichier de HUBIC 
			\item PFGE\_R\_IN : Entrée d'une réponse de HUBIC 
			\item PFGE\_R\_OUT : Sortie d'une réponse de HUBIC 
			\item <NOM TRAITEMENT> : Nom du traitement interne de HUBIC en cas d'erreur interne ou de point de passage intermédiaire 
		\end{itemize}
		\end{minipage}
&
	Consultation\\
	\hline\rowcolor{gris}
	Eventdatetime&Durée du traitement du flux&Date au format "hh :mm.ss.fff+hh :mm aaaa-mm-jj"&Consultation\\\hline
	\end{longtable}